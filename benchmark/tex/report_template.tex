\documentclass{article}

% Preamble {{{
\usepackage[a4paper, left=1.5in]{geometry}
\usepackage{graphicx}
\usepackage[utf8]{inputenc}
\usepackage{mdframed}
\usepackage{hyperref}
\usepackage{minted}
\renewcommand{\MintedPygmentize}{pygmentize}
\usepackage{verbatim}
\usepackage{amsmath,amssymb}
\usepackage{svg}

\newcommand{\li}{\item}
\newcommand{\lb}{\begin{itemize}}
\newcommand{\lbi}{\begin{itemize}\item}
\newcommand{\ls}{\begin{itemize}\item}
\renewcommand{\le}{\end{itemize}}
\renewcommand{\b}[1]{\textbf{#1}}
\renewcommand{\t}[1]{\texttt{#1}}
\renewcommand{\u}[2]{\href{#2}{#1}}
\newcommand{\st}[1]{\sout{#1}}

\newcommand{\SYSHASH}{628f56}
\newcommand{\TAG}{1003}
\newcommand{\bmimage}[1]{
    \begin{center}
    \includesvg[inkscapelatex=false,width=0.8\textwidth]{#1_\SYSHASH_\TAG}
    \end{center}}

\usepackage[T1]{fontenc}
\usepackage{stix}
% \usepackage{fontspec}
% \setmainfont{Asana-Math}
\usepackage{tgpagella}
\usepackage{tgtermes}
\usepackage{lmodern}

\graphicspath{ {../_img} }

% }}}

\begin{document}

\tableofcontents

\section{Preambule}

\begin{enumerate}
    \item We use the following problems for the benchmark:
        \begin{itemize}
            \item \t{Grover} - The grover-like problem parametrized by the \textbf{number of qubits} and the
                \textbf{number of layers}.
            \item \t{QFT-Hybrid} - The QFT-like problem called in a classical loop, parametrized by
                the \textbf{number of qubits} and the \textbf{number of quantum layers}.
            \item \t{ChemVQE} - VQE-like problem, for which the quantum gradient is called in a
                classical loop. Parametrized by the \textbf{number of qubits} and the
                \textbf{differentiation method} name.
        \end{itemize}
    \item For each problem, we vary the number of qubits, the number of layers, both
        number of qubits and the differentiation method (if applicable).
    \item For every problem, we try the following implementations:
        \begin{itemize}
            \item \t{catalyst/lighting}
            \item \t{pennylane/default.qubit}
            \item \t{pennylane/lightning}
            \item \t{pennylane+jax/default.qubit}
            \item \t{pennylane+jax/lightning}
        \end{itemize}
    \item For every implementation, we measure \t{compilation} and \t{running} times. We follow the
    framework authors' terminology, e.g. we define \t{PennyLane+Jax compilation} as calling the
    \t{compile} function of the JAX function object.

    \item Totally we run a number of different measurement procedures, as encoded in the
    \href{https://github.com/XanaduAI/pennylane-mlir/blob/benchmarking-1-2/benchmark/catalyst_benchmark/main.py#L84}{Python module}.

    \item For all the measurements we aim to put all the available data on the plots.
\end{enumerate}

\pagebreak
\section{Grover-like problem}

\subsection{Varying number of qubits}

\bmimage{regular_grover_compile}

\bmimage{regular_grover_runtime}

\pagebreak
\subsection{Varying number of layers}

\bmimage{deep_grover_compile}

\bmimage{deep_grover_runtime}

\section{Hybrid-QFT problem}

\subsection{Varying number of qubits}

\bmimage{regular_qfth_compile}

\bmimage{regular_qfth_runtime}

\pagebreak
\subsection{Varying number of layers}

\bmimage{deep_qfth_compile}

\bmimage{deep_qfth_runtime}

\pagebreak
\section{VQE-like problem}

\subsection{Varying number of qubit and gradient methods}

\subsubsection{As line-charts}

\bmimage{variational_compile_adjoint_lineplot}

\bmimage{variational_compile_backprop_lineplot}

\bmimage{variational_compile_finitediff_lineplot}

\bmimage{variational_compile_parametershift_lineplot}

\pagebreak

\bmimage{variational_runtime_adjoint_lineplot}

\bmimage{variational_runtime_backprop_lineplot}

\bmimage{variational_runtime_finitediff_lineplot}

\bmimage{variational_runtime_parametershift_lineplot}

\pagebreak
\subsubsection{As bar-charts}

\bmimage{variational_runtime_adjoint}

\bmimage{variational_runtime_backprop}

\bmimage{variational_runtime_finitediff}

\bmimage{variational_runtime_parametershift}

\pagebreak
\section{Appendinx: Trial views}

\subsection{Grover-like problem}
\bmimage{regular_grover_compile_trial}

\bmimage{regular_grover_runtime_trial}

\bmimage{deep_grover_compile_trial}

\bmimage{deep_grover_runtime_trial}

\subsection{Hybrid-QFT problem}

\bmimage{regular_qfth_compile_trial}

\bmimage{regular_qfth_runtime_trial}

\bmimage{deep_qfth_compile_trial}

\bmimage{deep_qfth_runtime_trial}

\subsection{VQE-like problem}

\bmimage{variational_compile_trial_adjoint_lineplot}

\bmimage{variational_compile_trial_backprop_lineplot}

\bmimage{variational_compile_trial_finitediff_lineplot}

\bmimage{variational_compile_trial_parametershift_lineplot}

\bmimage{variational_runtime_trial_adjoint_backprop_lineplot}

\bmimage{variational_runtime_trial_finitediff_lineplot}

\bmimage{variational_runtime_trial_parametershift_lineplot}

\end{document}
