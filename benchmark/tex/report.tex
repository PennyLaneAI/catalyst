\documentclass{article}

% Preamble {{{
\usepackage[a4paper, total={4in, 10in}]{geometry}
\usepackage{graphicx}
\usepackage[utf8]{inputenc}
\usepackage{mdframed}
\usepackage{hyperref}
\usepackage{minted}
\renewcommand{\MintedPygmentize}{pygmentize}
\usepackage{verbatim}
\usepackage{amsmath,amssymb}
\usepackage{svg}

\newcommand{\li}{\item}
\newcommand{\lb}{\begin{itemize}}
\newcommand{\lbi}{\begin{itemize}\item}
\newcommand{\ls}{\begin{itemize}\item}
\renewcommand{\le}{\end{itemize}}
\renewcommand{\b}[1]{\textbf{#1}}
\renewcommand{\t}[1]{\texttt{#1}}
\renewcommand{\u}[2]{\href{#2}{#1}}
\newcommand{\st}[1]{\sout{#1}}

\newcommand{\SYSHASH}{628f56}
\newcommand{\TAG}{1003}
\newcommand{\bmimage}[1]{
    \begin{center}
    \includesvg[inkscapelatex=false,width=0.8\textwidth]{#1_\SYSHASH_\TAG}
    \end{center}}

\usepackage[T1]{fontenc}
\usepackage{stix}
% \usepackage{fontspec}
% \setmainfont{Asana-Math}
\usepackage{tgpagella}
\usepackage{tgtermes}
\usepackage{lmodern}

\graphicspath{ {../_img} }

% }}}
\begin{document}

\tableofcontents

\section{Preambule}

\begin{enumerate}
    \item We benchmark the following problems:
        \begin{itemize}
            \item \t{Grover} ("regiular" circuit)
            \item \t{Grover} ("deep" circuit)
            \item \t{ChemVQE} (variational problem)
        \end{itemize}
    \item For every problem, we benchmark the follwing implementations:
        \begin{itemize}
            \item \t{catalyst/lighting}
            \item \t{pennylane/*}
            \item \t{pennylane+jax/*}
        \end{itemize}
    \item For every implementation, we measure \t{compilation} and \t{running} times. We follow the
    framework authors' terminology, e.g. we define \t{PennyLane+Jax compilation} as calling the
    \t{compile} function of the JAX function object.

    \item Totally we run six different measurement procedures encoded in the
    \href{https://github.com/XanaduAI/pennylane-mlir/blob/benchmarking-1-2/benchmark/catalyst_benchmark/main.py#L84}{Python module}.

    \item For all the measurements we aim to put all the available data on the plots, thus the
        additional "trial" dimention encoded as varying opacity.
\end{enumerate}

\pagebreak
\section{Regular circuits}

\bmimage{regular_compile}

\bmimage{regular_runtime}


\pagebreak
\section{Deep circuits}

\bmimage{deep_compile}

\bmimage{deep_runtime}

\pagebreak
\section{Variational circuits (as line plots)}

\bmimage{variational_compile_adjoint_lineplot}

Comment: We were able to compile our VQE problem using the Default qubit device with "adjoint"
differentiation method, but we were not able to run it (see the runtime measurement plot below).

\bmimage{variational_compile_backprop_lineplot}

\bmimage{variational_compile_finitediff_lineplot}

\bmimage{variational_compile_parametershift_lineplot}

\pagebreak

\bmimage{variational_runtime_adjoint_lineplot}

\bmimage{variational_runtime_backprop_lineplot}

\bmimage{variational_runtime_finitediff_lineplot}

\bmimage{variational_runtime_parametershift_lineplot}

\pagebreak
\section{Variational circuits (as bar charts)}

\bmimage{variational_runtime_adjoint}

\bmimage{variational_runtime_backprop}

\bmimage{variational_runtime_finitediff}

\bmimage{variational_runtime_parametershift}

Comment: the \t{PLjax/Def} green bar at 12 qubits is indeed not a timeout.

\pagebreak
\section{Appendinx: Trial views}

\bmimage{regular_compile_trial}

\bmimage{regular_runtime_trial}

\bmimage{deep_compile_trial}

\bmimage{deep_runtime_trial}

\bmimage{variational_compile_trial_adjoint_lineplot}

\bmimage{variational_compile_trial_backprop_lineplot}

\bmimage{variational_compile_trial_finitediff_lineplot}

\bmimage{variational_compile_trial_parametershift_lineplot}

\bmimage{variational_runtime_trial_adjoint_backprop_lineplot}

\bmimage{variational_runtime_trial_finitediff_lineplot}

\bmimage{variational_runtime_trial_parametershift_lineplot}

\end{document}
